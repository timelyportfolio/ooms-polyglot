Jeroen Ooms (\href{https://jeroenooms.github.io/}{@opencpu}) provides \texttt{R}
users a magical polyglot world of R, JavaScript, C, and C++. This is my
attempt to both thank him and highlight some of all that he has done.
Much of my new R depends on his work.

\subsection{Ooms\textquotesingle{} Packages}

\href{http://www.r-pkg.org/}{metacran} provides a list of all \href{http://www.r-pkg.org/maint/jeroen.ooms@stat.ucla.edu}{Jeroen\textquotesingle{}s CRAN
packages}. Now, I
wonder if any of his packages are in the \emph{Top Downloads}.

\subsubsection{jsonlite}

Let\textquotesingle{}s leverage the helpful meta again from
\href{http://www.r-pkg.org/}{metacran} and very quickly get some assistance
from \emph{hint-hint}
\href{https://cran.r-project.org/web/packages/jsonlite/index.html}{\texttt{jsonlite}}.

\begin{verbatim}
library(jsonlite)
library(formattable)
library(tibble)
library(magrittr)

fromJSON("http://cranlogs.r-pkg.org/top/last-month/9") %>%
  {as_tibble(rank=rownames(.$downloads),.$downloads)} %>%
  rownames_to_column(var = "rank") %>%
  format_table(
    formatters = list(
      area(row=which(.$package=="jsonlite")) ~ formatter("span", style="background-color:#D4F; width:100%")
    )
  )
\end{verbatim}

\subsubsection{V8}

\href{https://github.com/jeroenooms/V8}{\texttt{V8}} gives \texttt{R} its own embedded
JavaScript engine to leverage functionality in JavaScript that might not
exist in \texttt{R}. For example, the
\href{http://marvl.infotech.monash.edu/webcola}{\texttt{WebCola}} constraint-based
layout engine offers valuable technology not available within R. Let\textquotesingle{}s
partially recreate the \href{http://marvl.infotech.monash.edu/webcola/examples/smallgroups.html}{smallgroups
example}
all in R. You might notice that the previously mentioned \texttt{jsonlite} is
essential to this workflow.

\begin{verbatim}
library(V8)
library(jsonlite)
library(scales)

ctx = new_context(global="window")

ctx$source("https://cdn.rawgit.com/tgdwyer/WebCola/master/WebCola/cola.min.js")

## [1] "true"

### small grouped example
group_json <- fromJSON(
  system.file(
    "htmlwidgets/lib/WebCola/examples/graphdata/smallgrouped.json",
    package = "colaR"
  )
)

# need to get forEach polyfill
ctx$source(
  "https://cdnjs.cloudflare.com/ajax/libs/es5-shim/4.1.10/es5-shim.min.js"
)

# code to recreate small group example
js_group <- '
// console.assert does not exists
console = {}
console.assert = function(){};

var width = 960,
  height = 500

graph = {
"nodes":[
  {"name":"a","width":60,"height":40},
  {"name":"b","width":60,"height":40},
  {"name":"c","width":60,"height":40},
  {"name":"d","width":60,"height":40},
  {"name":"e","width":60,"height":40},
  {"name":"f","width":60,"height":40},
  {"name":"g","width":60,"height":40}
],
"links":[
  {"source":1,"target":2},
  {"source":2,"target":3},
  {"source":3,"target":4},
  {"source":0,"target":1},
  {"source":2,"target":0},
  {"source":3,"target":5},
  {"source":0,"target":5}
],
"groups":[
  {"leaves":[0], "groups":[1]},
  {"leaves":[1,2]},
  {"leaves":[3,4]}
  ]
}

var g_cola = new cola.Layout()
  .linkDistance(100)
  .avoidOverlaps(true)
  .handleDisconnected(false)
  .size([width, height]);

g_cola
  .nodes(graph.nodes)
  .links(graph.links)
  .groups(graph.groups)
  .start()
'

# run the small group JS code in V8
ctx$eval(js_group)

## [1] "[object Object]"
\end{verbatim}

Now, \texttt{WebCola} has done the hard work and laid out our nodes and links,
so let\textquotesingle{}s get their positions.

\begin{verbatim}
nodes <- ctx$get('
  graph.nodes.map(function(d){
    return {name: d.name, x: d.x, y: d.y, height: d.height, width: d.width};
  })
')

links <- ctx$get('
  graph.links.map(function(d){
    return {x1: d.source.x, y1: d.source.y, x2: d.target.x, y2: d.target.y}
  })
')
\end{verbatim}

Some great examples of packages employing \texttt{V8} are
\href{http://www.r-pkg.org/pkg/geojsonio}{\texttt{geojsonio}},
\href{https://github.com/ropensci/lawn}{\texttt{lawn}},
\href{http://www.r-pkg.org/pkg/DiagrammeRsvg}{\texttt{DiagrammeRsvg}},
\href{https://github.com/ateucher/rmapshaper}{\texttt{rmapshaper}}, and
\href{https://github.com/edwindj/daff}{\texttt{daff}}.

\subsubsection{rjade}

We got layout coordinates above. Let\textquotesingle{}s use another one of Jeroen\textquotesingle{}s
packages \href{https://github.com/jeroenooms/rjade}{\texttt{rjade}} that provides
\href{http://jade-lang.com/}{\texttt{jade}} (now called
\href{https://github.com/pugjs/pug}{pug}) templates through \texttt{V8}. \texttt{rjade}
will let us build a \texttt{SVG} graph with our layout.

\begin{verbatim}
library(rjade)
library(htmltools)

svg <- jade_compile(
'
doctype xml
svg(version="1.1",xmlns="http://www.w3.org/2000/svg",xmlns:xlink="http://www.w3.org/1999/xlink",width="960px",height="500px")
  each l in lines
    line(style={fill:none, stroke:"lightgray"})&attributes({"x1": l.x1, "x2": l.x2, "y1": l.y1, "y2": l.y2})
  each val in rects
    g
      rect(style={fill: fillColor})&attributes({"x": val.x - val.width/2, "y": val.y - val.height/2, "height": val.height - 6, "width": val.width - 6, rx: 5, ry: 5})
      text&attributes({"x": val.x, "y": val.y, "dy": ".2em", "text-anchor":"middle"})= val.name
'
      ,pretty=T
)(rects = nodes, lines = links, fillColor = "lightgray")

HTML(svg)
\end{verbatim}

If we are not in the browser though with inline \texttt{SVG} support, we very
likely will want a static image format such as \texttt{png} or \texttt{jpeg}. Of
course, Jeroen has that covered also with the crazy-speedy
\href{https://github.com/jeroenooms/rsvg}{\texttt{rsvg}}. Jeroen offers
\href{https://github.com/jeroenooms/base64}{\texttt{base64}}, but in this case we
will use \texttt{base64enc}, since it allows \texttt{raw}.

\begin{verbatim}
library(rsvg)
library(base64enc)

graph_png <- rsvg_png(charToRaw(svg))

tags$img(src=dataURI(graph_png), mime="image/png")
\end{verbatim}

\subsubsection{magick}

Jeroen\textquotesingle{}s newest package \href{https://github.com/jeroenooms/magick}{\texttt{magick}}
is in my mind the coolest. \texttt{magick} gives us all the power of
\href{http://www.imagemagick.org/script/index.php}{\texttt{ImageMagick}} as easy \texttt{R}
functions, and is pure wizardry. I am still shocked that it compiled
first try with absolutely no problems.

\begin{verbatim}
library(magick)

graph_img <- image_read(graph_png)
wizard_img <- image_read("http://www.imagemagick.org/image/wizard.png")

images <- image_annotate(
  image_append(
    c(
      image_scale(image_crop(wizard_img, "600x600+100+100"), "100"),
      image_crop(graph_img, "400x400+200+0")
    )
  ),
  "Ooms is a Wizard!",
  size = 20,
  color = "blue",
  location = "+100+200"
)

tags$img(src=dataURI(image_write(images)), mime="image/png")
\end{verbatim}
